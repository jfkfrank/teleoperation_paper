\documentclass[journal,twoside]{JoPhA}

\usepackage{flushend}
\usepackage{color}

\begin{document}

\title{Strategies for Haptic-Robotic Teleoperation in Board Games: Playing checkers with Baxter}

\author{Gonzalo Esteban, Francisco J. Rodr\'{i}guez--Sedano, Laura Inyesta, Pablo Blanco, Francisco J. Rodr\'{i}guez--Lera
\IEEEcompsocitemizethanks{\IEEEcompsocthanksitem Miguel Cazorla is with University of Alicante.\protect\\

E-mail: miguel@dccia.ua.es
\IEEEcompsocthanksitem Vicente Matell\'an is with University of Rey Juan Carlos.} % <-this % s
}


\markboth{Journal of Physical Agents,~Vol.~1, No.~1, July~2007}%
{Cazorla and Matellan : JoPhA Paper Demo}
\maketitle


\begin{abstract}
Teleoperating robots is quite a common practice in fields such as surgery, defence or rescue. The main source of information in this kind of environments is the sense of sight. The user can see on a display what the robot is watching in real time, and maybe also a visual representation of the robot surroundings. Our proposal involves the use of haptic devices to teleoperate a robot, Baxter, in order for the user to obtain touch feedback besides visual information. As a proof of concept, the proposed environment is checkers playing. Our goal is testing if the inclusion of the sense of touch improves the user experience or not. 
\end{abstract}


\begin{IEEEkeywords}
JoPhA, journal, \LaTeX, paper, template.
\end{IEEEkeywords}


\section{Introduction}
% GONZALO?

% Teleoperation-telemanipulation and user experience
  \IEEEPARstart{T}{eleoperating} robots is quite a common practice in many fields such as surgery, defence or rescue~\cite{Vertut85}. The reason is simple: assisting a person to perform and accomplish complex or uncertain tasks in some environments may mean the difference between failure or success.

  These systems use different interfaces (e.g. cameras, microphones or input devices) to provide sensory information to the operator, and thus, improving the user experience. Traditionally, video feedback from an on-board or front-mounted camera is limited by technical constraints~\cite{Woods04,Woods97} like a restricted field of view or poor resolution. In some scenarios, these constraints makes difficult for the operator to be aware of the robot's proximity to obstacles~\cite{Alfano90}, causing a decrease in performance. To alleviate such limitations, haptic cues can be used to ~\cite{}.
  
% Interfaces: haptics and delay
%...

% GOALS OF THE PAPER
%...

% STRUCTURE
This paper is organized as follows. In Section~\ref{}.~\ldots The paper ends with our conclusions.

\section{Environment description}
% FRAN? + GONZALO?

\subsection{Slave: Haptic Device}
Haptic feedback takes advantage of the human sense of touch by recreating forces, vibrations or motions to the user. 
An haptic device allows to feel a feedback from virtual or real objects and produces a realistic touch sensations to a user. 
The PHANToM Omni is a 6 DOF haptic device that applies haptic feedback. 

 

It has 3 actuated DOF associated to the armature which provides the translational movements (X, Y, and Z Cartesian coordinates) and other 3 non-actuated DOF associated to the gimball  that gives the orientation (pitch, roll and yaw rotational movements).  The characteristic of this device allow to offer a feedback up to a maximum of 3.3N.


\subsection{Master: Baxter}
Subsubsection text here.

\subsection{Controller}
{\color{red}{Hay que adaptar a nuestro sistema}}

To use this device for the six DOF robot, the robot was broken into two 
sets of three degrees of freedom.  The first system is determined by 
the Cartesian coordinates of the wrist joint.  

The  rotation  of  the  gimbal  is  then  used  to  correspond  to  the  three rotational  degrees  of  freedom  (Rx,  Ry,  and  Rz)  of  the  forearm  of  the  robot.    
By  doing  this,  the  problems  associated  with  multiple  solutions  to  larger  order  degree  of  freedom  robotic  systems  are  mitigated.    
It  is  approximated  that  the  three  axese intersect  at  the  wrist despite the length of the link between the lower arm and forearm.  
This is justified due to the relatively short length of this link and the fact that the robot is controlled based on the vision of a surgeon, allowing for intuitive compensation by the surgeon to position the  end  effector.
This  visual  compensation  is  also  used  to  mitigate  the  effects  of  motor backlash propagation through the robot, although solutions to reduce backlash are being attempted.

\section{Experiment}
% SEDANO ? + LAURA Y PABLO?

\subsection{Game description}

Checkers is a strategy gameboard for two players that play on opposite sides of board. It can be played on a 8x8, 10x10 and 12x12 checkerboards.
There are two kind of pieces: the dark pieces and the light pieces. 
The game consists on move a piece diagonally to an adjacent unoccupied square. If the adjacent square contains an opponent's piece, and the square immediately beyond it is vacant, the piece may be captured and take away from the board by jumping over it. During the game each player alternate turns.

\subsection{Board adaptation}


\section{Conclusion}
% CAMINO?

The conclusion goes here.


\section*{Acknowledgment}

This work is supported by the \textit{C\'{a}tedra Telef\'{o}nica--Universidad de Le\'{o}n} and it has been partially funded by Spanish \textit{Ministerio de Econom\'{i}a y Competitividad} under grant DPI2013-40534-R.

\begin{thebibliography}{1}

\bibitem{Vertut85}
J.~Vertut and P.~Coiffet, \emph{Teleoperation and Robotics. Applications and Technology}, Spring Netherlands . \hskip 1em plus 0.5em minus 0.4em\relax, Vol. 3B, 1985.

\bibitem{Woods04}
D.D.~Woods, J.~Tittle, M.~Feil and A.~Roesler, \emph{Envisioning human-robot coordination in future operations}, IEEE Transactions on Systems Man and Cybernetics: Part C (Applications and Reviews).\hskip 1em plus 0.5em minus 0.4em\relax, Vol. 34, no.2, pp. 210--218, 2004.

\bibitem{Woods97}
D.~Woods and J.~Watts, \emph{How not to have to navigate through too many displays}, In Handbook of Human-Computer Interaction.estricting the field of view: Perceptual
and performance effects.\hskip 1em plus 0.5em minus 0.4em\relax 2nd ed, M.~Helander, T.~Landauer and P~.Prabhu, Eds. Amsterdam, The Netherlands: Elsevier Science, pp. 1177--1201, 1997.

\bibitem{Alfano90}
P.L.~Alfano and G.~F.~Michel, \emph{Restricting the field of view: Perceptual and performance effects}, Perceptual and Motor Skills.\hskip 1em plus 0.5em minus 0.4em\relax, Vol. 70, no. 1, pp. 35--45, 1990.

\end{thebibliography}
\end{document}


