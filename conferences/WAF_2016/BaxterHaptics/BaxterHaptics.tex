\documentclass[journal,twoside]{JoPhA}

\usepackage{flushend}

\begin{document}

\title{JoPhA Paper Demo}

\author{Miguel Cazorla and Vicente Matell\'an
\IEEEcompsocitemizethanks{\IEEEcompsocthanksitem Miguel Cazorla is with University of Alicante.\protect\\

E-mail: miguel@dccia.ua.es
\IEEEcompsocthanksitem Vicente Matell\'an is with University of Rey Juan Carlos.} % <-this % s
}


\markboth{Journal of Physical Agents,~Vol.~1, No.~1, July~2007}%
{Cazorla and Matellan : JoPhA Paper Demo}
\maketitle


\begin{abstract}
The abstract goes here.
\end{abstract}


\begin{IEEEkeywords}
JoPhA, journal, \LaTeX, paper, template.
\end{IEEEkeywords}


\section{Introduction}

\IEEEPARstart{T}{his} demo file is intended to serve as a ``starter file''
for JoPhA journal papers produced under \LaTeX\ using
JoPhA.cls. JoPhA.cls is based on IEEEtran.cls style. 

I wish you the best of success.

\hfill mds
 
\hfill January 11, 2007

\subsection{Subsection Heading Here}
Subsection text here.


\subsubsection{Subsubsection Heading Here}
Subsubsection text here.


\section{Guidelines}
Some basic guidelines are given:
\begin{itemize}
\item Paper must be written in two columns.
\item Use the \emph{flushend} package to compensate the last page. 
\item All the main words in the title must be capitalized. 
\end{itemize}



\section{Conclusion}
The conclusion goes here.


\section*{Acknowledgment}
Acknowlegments allways in this sections.

The authors would like to thank...

\begin{thebibliography}{1}

\bibitem{IEEEhowto:kopka}
H.~Kopka and P.~W. Daly, \emph{A Guide to \LaTeX}, 3rd~ed.\hskip 1em plus
  0.5em minus 0.4em\relax Harlow, England: Addison-Wesley, 1999.

\end{thebibliography}
\end{document}


