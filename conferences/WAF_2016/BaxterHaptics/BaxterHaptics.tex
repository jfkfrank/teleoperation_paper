\documentclass[journal,twoside]{JoPhA}

\usepackage{flushend}
\usepackage{color}

\begin{document}

\title{Strategies for Haptic-Robotic Teleoperation in Board Games: Playing checkers with Baxter}

\author{Gonzalo Esteban, Francisco J. Rodr\'{i}guez--Sedano, Laura Inyesto, Pablo Blanco, Francisco J. Rodr\'{i}guez--Lera
\IEEEcompsocitemizethanks{\IEEEcompsocthanksitem Miguel Cazorla is with University of Alicante.\protect\\

E-mail: miguel@dccia.ua.es
\IEEEcompsocthanksitem Vicente Matell\'an is with University of Rey Juan Carlos.} % <-this % s
}


\markboth{Journal of Physical Agents,~Vol.~1, No.~1, July~2007}%
{Cazorla and Matellan : JoPhA Paper Demo}
\maketitle


\begin{abstract}
Teleoperating robots is quite a common practice in fields such as surgery, defence or rescue. The main source of information in this kind of environments is the sense of sight. The user can see on a display what the robot is watching in real time, and maybe also a visual representation of the robot surroundings. Our proposal involves the use of haptic devices to teleoperate a robot, Baxter, in order for the user to obtain haptic feedback besides visual information. As a proof of concept, the proposed environment is checkers playing. Our goal is testing if the inclusion of the sense of touch improves the user experience or not. 
\end{abstract}


\begin{IEEEkeywords}
JoPhA, journal, \LaTeX, paper, template.
\end{IEEEkeywords}


\section{Introduction}
% GONZALO?

% Teleoperation-telemanipulation and user experience
  \IEEEPARstart{T}{eleoperating} robots is quite a common practice in many fields such as surgery, defence or rescue~\cite{Vertut85}. The reason is simple: assisting a person to perform and accomplish complex or uncertain tasks in some environments may mean the difference between failure or success.

  Enhancing the user experience is a key aspect in teleoperation. These systems use different interfaces (e.g. cameras, microphones or input devices) to provide sensory information to the operator, and thus, improving the user experience. Traditionally, video feedback from an on-board or front-mounted camera is limited by technical constraints~\cite{Woods04,Woods97} like a restricted field of view or poor resolution. In some scenarios, these constraints makes difficult for the operator to be aware of the robot's proximity to objects~\cite{Alfano90}, causing a decrease in performance. To alleviate such limitations, at least partially, haptic cues (either by force or tactile feedback) have been shown to be useful in some applications~\cite{Son13,Sitti03,Diolaiti02}, especially when the operator performs manipulation tasks~\cite{King09,Kron04}.
  
%   haptics must play a superior role over vision or audio.
  In recent years, haptic devices are playing a superior role over vision or audio, particularly in applications that involve touch.%...

% Interfaces: haptics and delay
%...

% GOALS OF THE PAPER
  The motivation behind this paper is to test whether or not the sense of touch improves the user experience in teleoperation. To achieve this, we propose an experiment: teleoperate a robot for playing a board game. The scenario will have two players and one game. One player is located at the game's place while the other is away, teleoperating a robot which is ``physically'' located at that same place. The teleoperation system consists of a Geomagic Touch haptic interface (that acts as a master device) and a Baxter robot (which acts as the slave device). The board game chosen to play is ``checkers'', as it is...

% STRUCTURE
  This paper is organized as follows. Section~\ref{sec:environment} shows the architecture of the environment by describing the basic elements of the system. Section~\ref{sec:experiment} describes how the experiment has been carried out.~\ldots Finally, the paper ends with our conclusions.

\section{Environment description}
\label{sec:environment}

% FRAN? + GONZALO?

% TELEMANIPULATION SYSTEMS AND BILATERAL TELEOPERATION. Highlight bilateral teleoperation using position [Kuchenbecker 2006].

  A basic telemanipulation system consists of a master robot manipulated by an operator and a slave...
  

\subsection{Slave: Haptic Device}
Haptic feedback takes advantage of the human sense of touch by recreating forces, vibrations or motions to the user. 
An haptic device allows to feel a feedback from virtual or real objects and produces a realistic touch sensations to a user. 
The PHANToM Omni is a 6 DOF haptic device that applies haptic feedback. 

 

It has 3 actuated DOF associated to the armature which provides the translational movements (X, Y, and Z Cartesian coordinates) and other 3 non-actuated DOF associated to the gimball  that gives the orientation (pitch, roll and yaw rotational movements).  The characteristic of this device allow to offer a feedback up to a maximum of 3.3N.


\subsection{Master: Baxter}
Subsubsection text here.

\subsection{Controller}
{\color{red}{Hay que adaptar a nuestro sistema}}

To use this device for the six DOF robot, the robot was broken into two 
sets of three degrees of freedom.  The first system is determined by 
the Cartesian coordinates of the wrist joint.  

The  rotation  of  the  gimbal  is  then  used  to  correspond  to  the  three rotational  degrees  of  freedom  (Rx,  Ry,  and  Rz)  of  the  forearm  of  the  robot.    
By  doing  this,  the  problems  associated  with  multiple  solutions  to  larger  order  degree  of  freedom  robotic  systems  are  mitigated.    
It  is  approximated  that  the  three  axese intersect  at  the  wrist despite the length of the link between the lower arm and forearm.  
This is justified due to the relatively short length of this link and the fact that the robot is controlled based on the vision of a surgeon, allowing for intuitive compensation by the surgeon to position the  end  effector.
This  visual  compensation  is  also  used  to  mitigate  the  effects  of  motor backlash propagation through the robot, although solutions to reduce backlash are being attempted.

\section{Experiment}
\label{sec:experiment}
% SEDANO ? + LAURA Y PABLO?

\subsection{Game description}

Checkers is a strategy gameboard for two players that play on opposite sides of board. It can be played on a 8x8, 10x10 and 12x12 checkerboards.
There are two kind of pieces: the dark pieces and the light pieces. 
The game consists on move a piece diagonally to an adjacent unoccupied square. If the adjacent square contains an opponent's piece, and the square immediately beyond it is vacant, the piece may be captured and take away from the board by jumping over it. During the game each player alternate turns.

\subsection{Board adaptation}


\subsection{Evaluation}

For evaluation of our experiment we used the Guideline for Ergonomic Haptic Interaction Design (GEHID) developed by L. M. Muñoz, P. Ponsa, and A. Casals \cite{Munoz12}. The guide provides an approach that relates human factors to robotics technology and is based on measures that characterize the haptic interfaces, user’s capabilities and the objects to manipulate. We chose this guide as it is a method that aims to cover aspects of haptic interface design and human-robot interaction in order to improve the design and use of human-robot haptic interfaces in telerobotics applications.

The method to be followed for using the GEDIH guide consists in forming a focus group composed of experts and designers in order to: analyze the indicator, measure the indicator, obtain the GEDIH global evaluation index and finally offer improvement recommendations in the interface design. After the GEDIH validation a user’s experience test can be prepared in order to measure human-robot metrics (task effectiveness, efficiency and satisfaction)~\cite{Andonovski10}.

In a first step we detailed a set of selected indicators that provide a quantitative and/or qualitative measure of the perceived information by the user from the teleoperated environment. In this case we selected the following indicators:
\begin{itemize}
\item \textbf{ReactionForce/Moment}, this indicator measures the variation in the force or moment perceived when contacting with an object or exerting a force over it.
\item \textbf{Pressure}, in this case the variation in the force perceived under a contact with a surface unit it is measured.
\item \textbf{Rigidity}, measure the absence of displacement perceived when a force is exerted. 
\item \textbf{Weight/Inertia}, this indicator measures the resistance that is perceived by user when an object is held statically or is displaced from one place to another.
\item \textbf{Impulse/Collision}, in this case the indicator measures the variation of the linear momentum that happens when colliding with objects in the teleoperated environment.
\item \textbf{Vibration}, this indicator measures the variation in the position perceived when an object is manipulated by user.
\item \textbf{Geometric Properties}, in this case is necessary the perception of the size and shape of the manipulated objects in the teleoperated environment.
\item \textbf{Disposition}, it is also necessary to measure the perception of the position and orientation of objects.
\end{itemize}

In the next stage we identified two different ways of moving the game pieces; one is to drag the tab on the board and another is to raise the piece and move it to the target position. The next step is a tasks allocation for these ways of moving objects. 

In the first case, the sequence of basic task involved are:
\begin{enumerate}
\item Presence. In order to determine if the object is near the grasping tool.
\item Grasping. Is the task that allows the operator to grab an object.
\item Push. Is the task that allows the user to move a game piece by applying a force over it and move it on the board. 
\item Assembling. In order to put the object in the target position.
\item Grasping. In this case to release the object.
\item Presence. In order to release the object.
\end{enumerate}

In the second case, we identified the following basic tasks:
\begin{enumerate}
\item Presence. In order to determine if the object is near the grasping tool.
\item Grasping. Is the task that allows the operator to grab an object.
\item Translating. In order to move the object towards the target place.
\item Assembling. In order to put the object in the target position.
\item Grasping. In this case to release the object.
\item Presence. In order to release the object.
\end{enumerate}

In both cases we observed that previous to the first and last task, the movement of the teleoperated does not requires haptic feedback.

The third step is to establish a clear relationship between selected indicators and basic haptic tasks. Table \ref{tabla_1} show this relation relationship. The first column shows the basic tasks involved in our  experiment and the specific indicators are placed in the second column. But also we consider for the design of haptic interface that different tasks have different requirements, both the haptic device as the robot. For example, in the translating task we need game piece weight perception that gives haptic force sensor and grasping force provided by the corresponding robot sensor. Moreover, in our experiment, we use a visual feedback provided by two cameras in the robot to determine the position of the game pieces and the operator can determine the necessary directions and trajectories to perform the desired task. 



\begin{table}
\centering
\caption{\label{tabla_1}Basic tasks and GEHID indicators.}
\begin{tabular}{ll}
\hline 
\textbf{\footnotesize{}Basic task} & \textbf{\footnotesize{}Indicator}\tabularnewline
\hline 
{\footnotesize{}Presence} & {\footnotesize{}Collision on 3D movement}\tabularnewline
{\footnotesize{}Grasping} & {\footnotesize{}Rigidity on the the grasping tool}\tabularnewline
{\footnotesize{}Push} & {\footnotesize{}Vibration on 3D movement}\tabularnewline
{\footnotesize{}Translating} & {\footnotesize{}Weight on 3D movement}\tabularnewline
{\footnotesize{}Assembling} & {\footnotesize{}Collision on 3D movement}\tabularnewline
\hline 
\end{tabular}

\end{table}


\section{Conclusion}
% CAMINO?

The conclusion goes here.


\section*{Acknowledgment}

This work is supported by the \textit{C\'{a}tedra Telef\'{o}nica--Universidad de Le\'{o}n} and it has been partially funded by Spanish \textit{Ministerio de Econom\'{i}a y Competitividad} under grant DPI2013-40534-R.

\begin{thebibliography}{1}

\bibitem{Vertut85}
J.~Vertut and P.~Coiffet, \emph{Teleoperation and Robotics. Applications and Technology}, Spring Netherlands. \hskip 1em plus 0.5em minus 0.4em\relax Vol. 3B, 1985.

\bibitem{Woods04}
D.D.~Woods, J.~Tittle, M.~Feil and A.~Roesler, \emph{Envisioning human-robot coordination in future operations}, IEEE Transactions on Systems, Man and Cybernetics: Part C (Applications and Reviews).\hskip 1em plus 0.5em minus 0.4em\relax Vol. 34, no.2, pp. 210--218, 2004.

\bibitem{Woods97}
D.~Woods and J.~Watts, \emph{How not to have to navigate through too many displays}, In Handbook of Human-Computer Interaction.estricting the field of view: Perceptual
and performance effects.\hskip 1em plus 0.5em minus 0.4em\relax 2nd ed, M.~Helander, T.~Landauer and P~.Prabhu, Eds. Amsterdam, The Netherlands: Elsevier Science, pp. 1177--1201, 1997.

\bibitem{Alfano90}
P.L.~Alfano and G.~F.~Michel, \emph{Restricting the field of view: Perceptual and performance effects}, Perceptual and Motor Skills.\hskip 1em plus 0.5em minus 0.4em\relax Vol. 70, no. 1, pp. 35--45, 1990.

\bibitem{Son13}
H.I.~Son, A.~Franchi, L.L.~Chuang, J.~Kim, H.H.~Bulthoff and P.R.~Giordano, \emph{Human-Centered Design and Evaluation of Haptic Cueing for Teleoperation of Multiple Mobile Robots}, IEEE Transactions on Systems, Man and Cybernetics: Part B (Cybernetics).\hskip 1em plus 0.5em minus 0.4em\relax Vol. 43, no. 2, pp. 597--609, 2013.

\bibitem{Sitti03}
M.~Sitti and H.~Hashimoto, \emph{Teleoperated Touch Feedback From the Surfaces at the Nanoscale: Modeling and Experiments}, IEEE ASME Transactions on Mechatronics.\hskip 1em plus 0.5em minus 0.4em\relax Vol. 8, no. 2, pp. 287--298, 2003.

\bibitem{Diolaiti02}
N.~Diolaiti and C.~Melchiorri, \emph{Tele-Operation of a Mobile Robot Through Haptic Feedback}, In Proceedings of the IEEE International Workshop on Haptic Virtual Environments and Their Applications (HAVE 2002).\hskip 1em plus 0.5em minus 0.4em\relax, Ottawa, Ontario, Canada, 17-18 November, 2002.

\bibitem{King09}
C.H.~King, M.O.~Culjat, M.L.~Franco, C.E.~Lewis, E.P.~Dutson, W.S.~Grundfest and J.W.~Bisley, \emph{Tactile Feedback Induces Reduced Grasping Force in Robot-Assisted Surgery}, IEEE Transactions on Haptics.\hskip 1em plus 0.5em minus 0.4em\relax Vol. 2, no. 2, pp. 103--110, 2009.

\bibitem{Kron04}
A.~Kron, G.~Schmidt, B.~Petzold, M.I.~Zah, P.~Hinterseer and E.~Steinbach, \emph{Disposal of explosive ordnances by use of a bimanual haptic telepresence system}, In Proceedings of the IEEE International Conference on Robotics and Automation (ICRA 04).\hskip 1em plus 0.5em minus 0.4em\relax pp. 1968--1973, 2004.

\bibitem{Munoz12}
L.M.~Mu{\~{n}}oz, P.~Ponsa and A.~Casals, \emph{Design and Development of a Guideline for Ergonomic Haptic Interaction}, In Human-Computer Systems Interaction: Backgrounds and Applications 2: Part 2, Z.~Hippe and J.L.~Kulikowski and T.~Mroczek, Eds. Springer Berlin Heidelberg, pp. 15--19, 2012.

\bibitem{Andonovski10}
B.~Andonovski, P.~Ponsa and A.~Casals, \emph{Towards the development of a haptics guideline in human-robot systems}, Human System Interactions (HSI), 2010 3rd Conference on, Rzeszow, pp. 380--387, 2010.


\end{thebibliography}
\end{document}


