% This is LLNCS.DOC the documentation file of
% the LaTeX2e class from Springer-Verlag
% for Lecture Notes in Computer Science, version 2.4
% \documentclass{llncs}
\documentclass[citeauthoryear]{llncs}
\usepackage{llncsdoc}
%

\markboth{Orbione-Bot Team Description Paper}{Lera et al.}

\def\addtocontents#1#2{}%
\def\addcontentsline#1#2#3{}%
\def\markboth#1#2{}%

\begin{document}
\title{Orbione-Bot Team Description Paper }

\author{Francisco J. Rodr\'iguez Lera, Fernando Casado, Jes\'us Balsa, Vicente Matell\'an\inst{1} \and Francisco Mart\'in Rico\inst{2}}

\institute{Universidad de Le\'on, 24071, Le�n, Spain
\and
Universidad Rey Juan Carlos, Madrid, Spain}

\titlerunning  \toctitle
\authorrunning \tocauthor

\maketitle
%
\begin{abstract}
This paper aims to show the recent progress of our team to set up the robot RB1. 
It describes the development of our team for taking part of the RoboCup@home 2016 that will be performed in Leipzig.
We define a software architecture that is made up of the next main components: object recogntion, navigation, dialogue, manipulation and context awareness.
All these components have been tested in open demostrations in our lab. 
The motivation of this team is to generate more acceptable socially assistive robots.
The RoboCup bring us the opportunity to move our demostrations from the lab to external environments before to test our researches in real home environments.
\end{abstract}
%
\section{Introduction}

Watermelon Project team was originally created by the group of robotics at the University of
Le\'on in 2012. In 2013 it was expanded with the addition of Robotics Group at the Rey Juan
Carlos University ~\cite{martin2014myrabot}. 


\section{Background}

Our original goal was to prove that a robotic platform could be created  using low-cost materials.
We wanted to validate it as a starter kit for robotics researchers in competitions like RoCKIn or other
endeavours. The RoCKIn 2014 events allowed us to test our approach intensively, and introduce
several improvements and fixes in our robot. Finally, we have concluded that 
this robot can be used in the labs for researching and for analyzing semi-autonomous behaviours in 
real homes. 

MYRA (?Elderly and Augmented Reality? in Spanish) was the code name given to the original
project more than two years ago. The main goal was to build an assistance robotics platform
that would be suited to the needs of old people, such as helping them to take the right
medication at the right time. We intended to study the human-robot interaction using
augmented reality, and a target demographic with usually little experience with robots or
technology overall. The Watermelon team was formed at the end of 2013, in order to
participate in the RoCKIn@Home camp that was going to take place in Rome on January the
next year. Since the first moment, we focused our efforts on adding new software and
hardware solutions, or improving the existing ones, in order to adapt MYRA for the
competition. Thus the MYRABot platform was created, and it has been continuously reshaped
ever since.


\section{Research}

\subsection{Navigation}

\subsection{Context Awareness}

\subsection{}



\section{Contributions}


\section{Reusability}



\section{Conclusions and Future Work}

% % % % \medskip\noindent{\itshape Sample Output}
% % % % \begin{table}
% % % % \caption{Critical $N$ values}
% % % % \begin{center}
% % % % \renewcommand{\arraystretch}{1.4}
% % % % \setlength\tabcolsep{3pt}
% % % % \begin{tabular}{llllll}
% % % % \hline\noalign{\smallskip}
% % % % ${\mathrm M}_\odot$ & $\beta_{0}$ & $T_{\mathrm c6}$ & $\gamma$
% % % %   & $N_{\mathrm{crit}}^{\mathrm L}$
% % % %   & $N_{\mathrm{crit}}^{\mathrm{Te}}$\\
% % % % \noalign{\smallskip}
% % % % \hline
% % % % \noalign{\smallskip}
% % % %  30 & 0.82 & 38.4 & 35.7 & 154 & 320 \\
% % % %  60 & 0.67 & 42.1 & 34.7 & 138 & 340 \\
% % % % 120 & 0.52 & 45.1 & 34.0 & 124 & 370 \\
% % % % \hline
% % % % \end{tabular}
% % % % \end{center}
% % % % \end{table}
% % % % 
% % % % Before continuing your text you need an empty line. \dots
% % % % 
% % % % \vspace{3mm}
% % % % For further information you will find a complete description of
% % % % the tabular environment
% % % % on p.~62~ff. and p.~204 of the {\em \LaTeX{} User's Guide \& Reference
% % % % Manual\/} by Leslie Lamport.
% % % % %
% % % % \subsection{Tables Not Coded with \protect\LaTeX{}}
% % % % %
% % % % If you do not wish to code your table using \LaTeX{}
% % % % but prefer to have it reproduced separately,
% % % % proceed as for figures and use the following coding:\\[2mm]
% % % % {\itshape Sample Input}
% % % % \begin{verbatim}
% % % % \begin{table}
% % % % \caption{text of your caption}
% % % % \vspace{x cm}     % the actual height needed for your table
% % % % \end{table}
% % % % \end{verbatim}
% % % % %
% % % % \subsection{Signs and Characters}
% % % % %
% % % % \subsubsection*{Special Signs.}
% % % % %
% % % % You may need to use special signs.  The available ones are listed in the
% % % % {\em \LaTeX{} User's Guide \& Reference Manual\/} by Leslie Lamport,
% % % % pp.~41\,ff.
% % % % We have created further symbols for math mode (enclosed in \$):
% % % % \begin{center}
% % % % \begin{tabular}{l@{\hspace{1em}yields\hspace{1em}}
% % % % c@{\hspace{3em}}l@{\hspace{1em}yields\hspace{1em}}c}
% % % % \verb|\grole| & $\grole$ & \verb|\getsto| & $\getsto$\\
% % % % \verb|\lid|   & $\lid$   & \verb|\gid|    & $\gid$
% % % % \end{tabular}
% % % % \end{center}
% % % % %
% % % % \subsubsection*{Gothic (Fraktur).}
% % % % %
% % % % If gothic letters are {\itshape necessary}, please use those of the
% % % % relevant \AmSTeX{} alphabet which are available using the amstex
% % % % package of the American Mathematical Society.
% % % % 
% % % % In \LaTeX{} only the following gothic letters are available:
% % % % \verb|$\Re$| yields $\Re$ and \verb|$\Im$| yields $\Im$. These should
% % % % {\itshape not\/} be used when you need gothic letters for your contribution.
% % % % Use \AmSTeX{} gothic as explained above. For the real and the imaginary
% % % % parts of a complex number within math mode you should use instead:
% % % % \verb|$\mathrm{Re}$| (which yields Re) or \verb|$\mathrm{Im}$| (which
% % % % yields Im).
% % % % %
% % % % \subsubsection*{Script.}
% % % % %
% % % % For script capitals use the coding
% % % % \begin{center}
% % % % \begin{tabular}{l@{\hspace{1em}which yields\hspace{1em}}c}
% % % % \verb|$\mathcal{AB}$| & $\mathcal{AB}$
% % % % \end{tabular}
% % % % \end{center}
% % % % (see p.~42 of  the \LaTeX{} book).
% % % % %
% % % % \subsubsection*{Special Roman.}
% % % % %
% % % % If you need other symbols than those below, you could use
% % % % the blackboard bold characters of \AmSTeX{},  but there might arise
% % % % capacity problems
% % % % in loading additional \AmSTeX{} fonts. Therefore  we created
% % % % the blackboard bold characters listed below.
% % % % Some of them are not esthetically
% % % % satisfactory. This need not deter you from using them:
% % % % in the final printed form they will be
% % % % replaced by the well-designed MT (monotype) characters of
% % % % the phototypesetting machine.
% % % % \begin{flushleft}
% % % % \begin{tabular}{@{}ll@{ yields }
% % % % c@{\hspace{1.em}}ll@{ yields }c}
% % % % \verb|\bbbc| & (complex numbers)   & $\bbbc$
% % % %   & \verb|\bbbf| & (blackboard bold F) & $\bbbf$\\
% % % % \verb|\bbbh| & (blackboard bold H) & $\bbbh$
% % % %   & \verb|\bbbk| & (blackboard bold K) & $\bbbk$\\
% % % % \verb|\bbbm| & (blackboard bold M) & $\bbbm$
% % % %   & \verb|\bbbn| & (natural numbers N) & $\bbbn$\\
% % % % \verb|\bbbp| & (blackboard bold P) & $\bbbp$
% % % %   & \verb|\bbbq| & (rational numbers)  & $\bbbq$\\
% % % % \verb|\bbbr| & (real numbers)      & $\bbbr$
% % % %   & \verb|\bbbs| & (blackboard bold S) & $\bbbs$\\
% % % % \verb|\bbbt| & (blackboard bold T) & $\bbbt$
% % % %   & \verb|\bbbz| & (whole numbers)     & $\bbbz$\\
% % % % \verb|\bbbone| & (symbol one)      & $\bbbone$
% % % % \end{tabular}
% % % % \end{flushleft}
% % % % \begin{displaymath}
% % % % \begin{array}{c}
% % % % \bbbc^{\bbbc^{\bbbc}} \otimes
% % % % \bbbf_{\bbbf_{\bbbf}} \otimes
% % % % \bbbh_{\bbbh_{\bbbh}} \otimes
% % % % \bbbk_{\bbbk_{\bbbk}} \otimes
% % % % \bbbm^{\bbbm^{\bbbm}} \otimes
% % % % \bbbn_{\bbbn_{\bbbn}} \otimes
% % % % \bbbp^{\bbbp^{\bbbp}}\\[2mm]
% % % % \otimes
% % % % \bbbq_{\bbbq_{\bbbq}} \otimes
% % % % \bbbr^{\bbbr^{\bbbr}} \otimes
% % % % \bbbs^{\bbbs_{\bbbs}} \otimes
% % % % \bbbt^{\bbbt^{\bbbt}} \otimes
% % % % \bbbz \otimes
% % % % \bbbone^{\bbbone_{\bbbone}}
% % % % \end{array}
% % % % \end{displaymath}
% % % % %
% % % % \section{References}
% % % % \label{refer}
% % % % %
% % % % There are three reference systems available; only one, of course,
% % % % should be used for your contribution. With each system (by
% % % % number only, by letter-number or by author-year) a reference list
% % % % containing all citations in the
% % % % text, should be included at the end of your contribution placing the
% % % % \LaTeX{} environment \verb|thebibliography| there.
% % % % For an overall information on that environment
% % % % see the {\em \LaTeX{} User's Guide \& Reference
% % % % Manual\/} by Leslie Lamport, p.~71.
% % % % 
% % % % There is a special {\sc Bib}\TeX{} style for LLNCS that works along
% % % % with the class: \verb|splncs.bst|
% % % % -- call for it with a line \verb|\bibliographystyle{splncs}|.
% % % % If you plan to use another {\sc Bib}\TeX{} style you are customed to,
% % % % please specify the option \verb|[oribibl]| in the
% % % % \verb|documentclass| line, like:
% % % % \begin{verbatim}
% % % % \documentclass[oribibl]{llncs}
% % % % \end{verbatim}
% % % % This will retain the original \LaTeX{} code for the bibliographic
% % % % environment and the \verb|\cite| mechanism that many {\sc Bib}\TeX{}
% % % % applications rely on.
% % % % %
% % % % \subsection{References by Letter-Number or by Number Only}
% % % % %
% % % % References are cited in the text -- using the \verb|\cite|
% % % % command of \LaTeX{} -- by number or by letter-number in square
% % % % brackets, e.g.\ [1] or [E1, S2], [P1], according to your use of the
% % % % \verb|\bibitem| command in the \verb|thebibliography| environment. The
% % % % coding is as follows: if you choose your own label for the sources by
% % % % giving an optional argument to the \verb|\bibitem| command the citations
% % % % in the text are marked with the label you supplied. Otherwise a simple
% % % % numbering is done, which is preferred.
% % % % \begin{verbatim}
% % % % The results in this section are a refined version
% % % % of \cite{clar:eke}; the minimality result of Proposition~14
% % % % was the first of its kind.
% % % % \end{verbatim}
% % % % The above input produces the citation: ``\dots\ refined version of
% % % % [CE1]; the min\-i\-mality\dots''. Then the \verb|\bibitem| entry of
% % % % the \verb|thebibliography| environment should read:
% % % % \begin{verbatim}
% % % % \begin{thebibliography}{[MT1]}
% % % % .
% % % % .
% % % % \bibitem[CE1]{clar:eke}
% % % % Clarke, F., Ekeland, I.:
% % % % Nonlinear oscillations and boundary-value problems for
% % % % Hamiltonian systems.
% % % % Arch. Rat. Mech. Anal. 78, 315--333 (1982)
% % % % .
% % % % .
% % % % \end{thebibliography}
% % % % \end{verbatim}
% % % % The complete bibliography looks like this:
% % % % %
% % % % \begin{thebibliography}{[MT1]}
% % % % %
% % % % \bibitem[CE1]{clar:eke}
% % % % Clarke, F., Ekeland, I.:
% % % % Nonlinear oscillations and
% % % % boundary-value problems for Hamiltonian systems.
% % % % Arch. Rat. Mech. Anal. 78, 315--333 (1982)
% % % % %
% % % % \bibitem[CE2]{clar:eke:2}
% % % % Clarke, F., Ekeland, I.:
% % % % Solutions p\'{e}riodiques, du
% % % % p\'{e}riode donn\'{e}e, des \'{e}quations hamiltoniennes.
% % % % Note CRAS Paris 287, 1013--1015 (1978)
% % % % %
% % % % \bibitem[MT1]{mich:tar}
% % % % Michalek, R., Tarantello, G.:
% % % % Subharmonic solutions with prescribed minimal
% % % % period for nonautonomous Hamiltonian systems.
% % % % J. Diff. Eq. 72, 28--55 (1988)
% % % % %
% % % % \bibitem[Ta1]{tar}
% % % % Tarantello, G.:
% % % % Subharmonic solutions for Hamiltonian
% % % % systems via a $\bbbz_{p}$ pseudoindex theory.
% % % % Annali di Matematica Pura (to appear)
% % % % %
% % % % \bibitem[Ra1]{rab}
% % % % Rabinowitz, P.:
% % % % On subharmonic solutions of a Hamiltonian system.
% % % % Comm. Pure Appl. Math. 33, 609--633 (1980)
% % % % \end{thebibliography}
% % % % %
% % % % \subsubsection*{Number-Only System.}
% % % % %
% % % % For this preferred system do not use the optional argument
% % % % in the \verb|\bibitem| command: then, only numbers will
% % % % appear for the citations in the text (enclosed in square brackets)
% % % % as well as for the marks in your
% % % % bibliography (here the number is only end-punctuated without
% % % % square brackets).
% % % % 
% % % % Subsequent citation numbers in the text are collapsed to ranges.
% % % % Non-numeric and undefined labels are handled correctly but no sorting is
% % % % done.
% % % % 
% % % % E.g., \verb|\cite{n1,n3,n2,n3,n4,n5,foo,n1,n2,n3,?,n4,n5}| -- where
% % % % \verb|n|$x$ is the key of the $x^{\mathrm{th}}$ \verb|\bibitem|
% % % % command in sequence, \verb|foo| is the key of a \verb|\bibitem| with an
% % % % optional argument, and \verb|?| is an undefined reference -- gives
% % % % 1,3,2-5,foo,1-3,?,4,5 as the citation reference.
% % % % 
% % % % \begin{verbatim}
% % % % \begin{thebibliography}{1}
% % % % \bibitem {clar:eke}
% % % % Clarke, F., Ekeland, I.:
% % % % Nonlinear oscillations and boundary-value problems for
% % % % Hamiltonian systems.
% % % % Arch. Rat. Mech. Anal. 78, 315--333 (1982)
% % % % \end{thebibliography}
% % % % \end{verbatim}
% % % % %
% % % % \subsection{Author-Year System}
% % % % %
% % % % References are cited in the text by name and year in parentheses
% % % % and should look as follows:
% % % % (Smith 1970, 1980), (Ekeland et al. 1985, Theorem 2), (Jones and Jaffe
% % % % 1986; Farrow 1988, Chap.\,2). If the name is part of the sentence
% % % % only the year may appear in parentheses,
% % % % e.g.\ Ekeland et al. (1985, Sect.\,2.1)
% % % % The reference list should contain all citations occurring in the text,
% % % % ordered alphabetically by surname (with initials following). If there
% % % % are several works by the same author(s) the references should be listed
% % % % in the appropriate order indicated below:
% % % % \begin{alpherate}
% % % % \setlength{\hfuzz}{5pt}
% % % % \item
% % % % One author: list works chronologically;
% % % % \item
% % % % Author and same co-author(s): list works chronologically;
% % % % \item
% % % % Author and different co-authors: list works alphabetically
% % % % according to co-authors.
% % % % \end{alpherate}
% % % % If there are several works by the same author(s) and in the same year,
% % % % but which are cited separately, they should be distinguished by the use
% % % % of ``a'', ``b'' etc., e.g.\ (Smith 1982a), (Ekeland et al. 1982b).
% % % % %
% % % % \subsubsection*{How to Code Author-Year System.}
% % % % %
% % % % If you want to use this system you have to specify the option
% % % % \verb|[citeauthoryear]| in the \verb|documentclass|, like:
% % % % \begin{verbatim}
% % % % \documentclass[citeauthoryear]{llncs}
% % % % \end{verbatim}
% % % % Write your citations in the text explicitly except for the year, leaving
% % % % that up to \LaTeX{} with the \verb|\cite| command. Then give only the
% % % % appropriate year as the optional argument (i.e. the label in square
% % % % brackets) with the \verb|\bibitem| command(s).\\[2mm]
% % % % {\itshape Sample Input}
% % % % \begin{verbatim}
% % % % The results in this section are a refined version
% % % % of Clarke and Ekeland (\cite{clar:eke}); the minimality result of
% % % % Proposition~14 was the first of its kind.
% % % % \end{verbatim}
% % % % The above input produces the citation: ``\dots\ refined version of
% % % % Clarke and Ekeland (1982); the minimality\dots''. Then the
% % % % \verb|\bibitem| entry of \verb|clar:eke| in the \verb|thebibliography|
% % % % environment should read:
% % % % \begin{verbatim}
% % % % \begin{thebibliography}{}  % (do not forget {})
% % % % .
% % % % .
% % % % \bibitem[1982]{clar:eke}
% % % % Clarke, F., Ekeland, I.:
% % % % Nonlinear oscillations and boundary-value problems for
% % % % Hamiltonian systems.
% % % % Arch. Rat. Mech. Anal. 78, 315--333 (1982)
% % % % .
% % % % .
% % % % \end{thebibliography}
% % % % \end{verbatim}
% % % % {\itshape Sample Output}

\bibliographystyle{splncs03}
% argument is your BibTeX string definitions and bibliography database(s)
\bibliography{biblio}
% \bibauthoryear
%
\newpage

\section{Team Description}

\begin{itemize}

  \item Team name: RB1
  \item Contact information: 
  \item Website: robotica.unileon.es
  \item Team members:
      \begin{itemize}
	\item Francisco Lera
	\item Fernando Garc�a
	\item Jes�s Balsa
	\item Vicente Matell�n
	\item Franciso Mart�n
      \end{itemize}
  
\end{itemize}

\section{Hardware Description}

\begin{itemize}

  \item Robot name: RB1
  \begin{enumerate}
    \item Base:
    
    \item Vision Sensors: RGB-D Xtion
    \item Range sensors: Frontal laser 
    \item Dialogue Sensors: Directional microphone Rode Rycote , Loudspeakers
    \item Manipulation: 7DOF arm. Dynamixel Pro servos. Workspace: 70 cm
    \item Computer Description: Intel Core i7 with 8 GB of RAM and a 200 GB of disk
    \item Torso: +40
    \item Head: pan-tilt head.
  \end{enumerate}

\end{itemize}


\section{Software Description}

\begin{itemize}
 \item Development framework: ROS
  \begin{enumerate}
   \item Paquete
   \item Paquete
   \item Paquete
   \item Paquete
   \item Paquete
   \item Paquete
  \end{enumerate}


  \item Robot Control: Bica 
\end{itemize}




\end{document}
